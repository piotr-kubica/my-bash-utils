\documentclass[8pt,landscape]{article}
\usepackage{multicol}
\usepackage{calc}
\usepackage{ifthen}
\usepackage[landscape]{geometry}
\usepackage{amsmath,amsthm,amsfonts,amssymb}
\usepackage{color,graphicx,overpic}
\usepackage{hyperref}
\usepackage{listings}
\usepackage{etoolbox}
\usepackage{fancyvrb}
\usepackage{array}
\usepackage[most]{tcolorbox}

\definecolor{bg}{RGB}{240,240,240}

% ---------------------------------------------------------

% reduce font size in Bib
\apptocmd{\thebibliography}{\fontsize{6}{7}\selectfont}{}{}%

% reduce line spacing in Bib
\let\OLDthebibliography\thebibliography
\renewcommand\thebibliography[1]{
  \OLDthebibliography{#1}
  \setlength{\parskip}{0pt}
  \setlength{\itemsep}{0pt plus 0.3ex}
}


\pdfinfo{
  /Title (example.pdf)
  /Creator (TeX)
  /Producer (pdfTeX 1.40.0)
  /Author (piku)
  /Subject (Bash Cheatsheet)
  /Keywords (pdflatex, latex,pdftex,tex)}

% This sets page margins to .5 inch if using letter paper, and to 1cm
% if using A4 paper. (This probably isn't strictly necessary.)
% If using another size paper, use default 1cm margins.
\ifthenelse{\lengthtest { \paperwidth = 11in}}
    { \geometry{top=.5in,left=.5in,right=.5in,bottom=.5in} }
    {\ifthenelse{ \lengthtest{ \paperwidth = 297mm}}
        {\geometry{top=1cm,left=1cm,right=1cm,bottom=1cm} }
        {\geometry{top=1cm,left=1cm,right=1cm,bottom=1cm} }
    }

% Turn off header and footer
\pagestyle{empty}

% Redefine section commands to use less space
\makeatletter
\renewcommand{\section}{\@startsection{section}{1}{0mm}%
                                {-1ex plus -.5ex minus -.2ex}%
                                {0.5ex plus .2ex}%x
                                {\normalfont\large\bfseries}}
\renewcommand{\subsection}{\@startsection{subsection}{2}{0mm}%
                                {-1explus -.5ex minus -.2ex}%
                                {0.5ex plus .2ex}%
                                {\normalfont\normalsize\bfseries}}
\renewcommand{\subsubsection}{\@startsection{subsubsection}{3}{0mm}%
                                {-1ex plus -.5ex minus -.2ex}%
                                {1ex plus .2ex}%
                                {\normalfont\small\bfseries}}
\makeatother

% Define BibTeX command
\def\BibTeX{{\rm B\kern-.05em{\sc i\kern-.025em b}\kern-.08em
    T\kern-.1667em\lower.7ex\hbox{E}\kern-.125emX}}

% Don't print section numbers
\setcounter{secnumdepth}{0}
\setlength{\parindent}{0pt}
\setlength{\parskip}{0pt plus 0.5ex}

% ---------------------------------------------------------

\begin{document}
\lstset{language=Bash}
\raggedright
\small
\begin{multicols}{3}

% multicol parameters
\setlength{\premulticols}{1pt}
\setlength{\postmulticols}{1pt}
\setlength{\multicolsep}{1pt}
\setlength{\columnsep}{2pt}

\section{Bash Cheatsheet}
\subsection{\textbf{awk 'pattern \{action\}' input.txt}}
\vspace{0.2cm}
Print 5th column\dots \\
\texttt{
awk '\{ print \$5 \}' input.txt
}
\\ \vspace{0.2cm}

\dots separated by comma \\
\texttt{
awk -F, '\{ print \$5 \}' input.txt
}
\\ \vspace{0.2cm}

\dots when 7th column equals \$7.30 \\
\texttt{
awk '\$7=="\textbackslash{}\$7.30" \{ print \$5 \}' input.txt
}
\\ \vspace{0.2cm}

\dots (regex) when line starts with 'st', ends with 'end' with chars between 'a' to 'z' \\
\texttt{
awk '/\textasciicircum{}st[a-z]end\$/ \{print \$5\}' input.txt
}
\\ \vspace{0.2cm}

Print sum of 2nd and 3rd column \\
\texttt{
awk '{print (\$2 + \$3)}' input.txt
}
\\ \vspace{0.2cm}

Print sum of all columns (\texttt{NF} holds column count) \\
\texttt{
awk '\{sum=0; for (col=1; col<=NF; col++) sum += \$col; print sum;\}' input.txt
}
\\ \vspace{0.2cm}

Print the sum of all rows (\texttt{END} terminates row) \\
\texttt{
awk '\{s += \$1\} END \{print s\}' input.txt
}

\vspace{0.2cm}
\subsection{\textbf{sort [options] [file\dots]}}
\vspace{0.2cm}
Sort by 5th column \\
\texttt{
sort -k5 input.txt
}
\\ \vspace{0.2cm}
Sort comma delimited by 5th col (\textit{-n} gives numerical sort) \\
\texttt{
sort -t, -nk5 user.csv
}


\vspace{0.2cm}
\subsection{\textbf{uniq [options] [file\dots]}}
\textit{Reads standard input comparing \textbf{adjacent lines}, and writes
  a copy of each unique input line to the standard output. \textbf{TIP: use with} \texttt{sort}} \\
\vspace{0.2cm}
Filter out duplicates \\
\texttt{
sort input.txt | uniq
}

\vspace{0.2cm}
Count times a line occures \\
\texttt{
sort input.txt | uniq -c
}

\vspace{0.2cm}
Print only duplicated lines \\
\texttt{
sort input.txt | uniq -d
}

\vspace*{\fill}
\columnbreak

\vspace{0.2cm}
\subsection{\textbf{sed [options] [file\dots]}}
\vspace{0.2cm}
Replace '4.5' with 'abc' \\
\texttt{
sed 's/4.5/abc/' input.txt
}
\vspace{0.2cm}
Filter lines containng 'John' \\
\texttt{
sed '/John/p' input.txt
}
\vspace{0.2cm}
Filter lines which do not contain 'John' \\
\texttt{
sed '/John/d' intput.txt
}
Filter lines 1 to 4 \\
\texttt{
sed '1-4d' intput.txt
}

\vspace{0.2cm}
\subsection{\textbf{xargs [options] [command]}}
\vspace{0.2cm}

Apply entire input as args \\
\texttt{
cat input.txt | xargs
} \\
\vspace{0.2cm}
\textit{input.txt} \begin{tcolorbox}[enhanced jigsaw,colback=bg,boxrule=0pt,arc=0pt]
one two \\
three
\end{tcolorbox}
output \begin{tcolorbox}[enhanced jigsaw,colback=bg,boxrule=0pt,arc=0pt]
one two three
\end{tcolorbox}
\vspace{0.2cm}

Apply arguments split by whitechars \texttt{-n} \\
\texttt{
cat input.txt | xargs -n 1
} \\
\vspace{0.2cm}
\textit{input.txt} \begin{tcolorbox}[enhanced jigsaw,colback=bg,boxrule=0pt,arc=0pt]
one two \\
three
\end{tcolorbox}
output \begin{tcolorbox}[enhanced jigsaw,colback=bg,boxrule=0pt,arc=0pt]
one \\ two \\ three
\end{tcolorbox}

Apply arguments split by line \texttt{-L} \\
\texttt{
cat input.txt | xargs -L 1
} \\
\vspace{0.2cm}
\textit{input.txt} \begin{tcolorbox}[enhanced jigsaw,colback=bg,boxrule=0pt,arc=0pt]
one two \\ three
\end{tcolorbox}
output \begin{tcolorbox}[enhanced jigsaw,colback=bg,boxrule=0pt,arc=0pt]
one two \\ three
\end{tcolorbox}

\vspace*{\fill}
\columnbreak


\vspace{0.2cm}
\subsection{\textbf{find [-H] [-L] [-P] [path\dots] [expression]}}
\vspace{0.2cm}
Find matching *.java files in current directory \\
\texttt{
find . -name "*.java"
}

\vspace{0.2cm}
Find matching *.java files in current directory \\
\texttt{
find . -name "*.java"
}

\vspace{0.2cm}
Find case-insensitive matching *.java files \\
\texttt{
find . -iname "*.java"
}


\vspace{0.2cm}
Find file matching Regex pattern files (\texttt{-iregex} case-insensitive) \\
\texttt{
find . -regex ".*jav."
}
-regex pattern

\vspace{0.2cm}
Find files modified 1 day ago (\texttt{-1}/\texttt{+1} for days less/more) \\
\texttt{
find . -mtime 1
}

\vspace{0.2cm}
Find files with permission 644 \\
\texttt{
find . -perm 644
}

\vspace{0.2cm}
Carry out a \textit{command} on each file that find matches \\
\texttt{
find . -regex '*.java' -exec \textit{wc -c} \{\} \textbackslash{};
}

\vspace{0.2cm}
\subsection{\textbf{grep [OPTIONS] PATTERN [FILE...]}}
\vspace{0.2cm}
Search for 'text' in input.txt file (regex) \\
\texttt{grep "text" input.txt} \\
\texttt{grep "t.*[x|X]t" input.txt}

\vspace{0.2cm}
Search for case-insensitive 'text' in matching files in*t.txt file \\
\texttt{
grep -i "text" in*t.txt
}

\vspace{0.2cm}
Display 3 lines before/after/around the match using \texttt{-B}/\texttt{-A}/\texttt{-C} \\
\texttt{
grep \textbf{-A} 3 -i "example" input.txt
}

\rule{0.3\linewidth}{0.25pt}
\footnotesize

\begin{thebibliography}{99}

\bibitem{MAN} https://linux.die.net/man/
\bibitem{TGS} https://www.thegeekstuff.com/2009/03/15-practical-unix-grep-command-examples
\bibitem{LW1} https://www.lifewire.com/write-awk-commands-and-scripts-2200573
\bibitem{LW2} https://www.lifewire.com/example-uses-of-sed-2201058
\bibitem{MBO} http://www.mblog.boo.pl/artykul-162-xargs-przesylanie-starnardowego-wejscia-jako-parametry-do-programu.html

\end{thebibliography}

\end{multicols}
\end{document}